
%MATH CHARACTER COMMANDS: Short hand for commands
\def\N{\mathbb{N}}		 				%Natual Bold Face: \N is now the command for the Natural Numbers
\def\Q{\mathbb{Q}} 						%Rational Bold Face: \R
\def\R{\mathbb{R}} 						%Real Bold Face
\def\Z{\mathbb{Z}} 						%Integers Bold Face
\def\C{\mathbb{C}} 						%Complex Bold Face
\def\eps{\varepsilon}                                               % Defines \varepsilon to \eps
\renewcommand{\qedsymbol}{$\blacksquare$} % QED symbol

\def\L{\mathcal{L}}
\def\boldy{\boldsymbol{y}}
\def\boldx{\boldsymbol{x}}
\def\boldF{\boldsymbol{F}}
\def\boldu{\boldsymbol{u}}
\def\boldv{\boldsymbol{v}}
\def\boldn{\boldsymbol{n}}
\def\boldN{\boldsymbol{N}}
\def\boldU{\boldsymbol{U}}
\def\boldR{\boldsymbol{R}}
\def\boldr{\boldsymbol{r}}
\def\boldq{\boldsymbol{q}}
\def\boldi{\boldsymbol{i}}

\def\boldalpha{\boldsymbol{\alpha}}
\def\boldtheta{\boldsymbol{\theta}}
\def\boldxi{\boldsymbol{\xi}}

\def\diag{\text{diag}}
\def\fluxone{\widehat{\widetilde{\boldsymbol{F}}}_1}
\def\fluxtwo{\widehat{\widetilde{\boldsymbol{F}}}_2}
\def\dt{\Delta t}
\def\jacobian{(\nabla_{\boldy} \boldf_n)}
\def\hessian{\left(\boldsymbol{H}_{\boldy}(\boldf)\right)_n}
\def\dtjacobian{ \frac{\left(\nabla_{\boldy} \boldf_n\right) }{\partial t} }
\newcommand*\widefbox[1]{\fbox{\hspace{2em}#1\hspace{2em}}}
\DeclareMathOperator{\arcsinh}{arcsinh}

%Solution template with little sideways triangle
\declaretheoremstyle[
spaceabove=6pt, spacebelow=6pt,
headfont=\normalfont\bfseries,
notefont=\mdseries, notebraces={(}{)},
bodyfont=\normalfont,
postheadspace=1em
]{exstyle}

\declaretheoremstyle[
spaceabove=6pt, spacebelow=6pt,
headfont=\normalfont\bfseries,
notefont=\mdseries, notebraces={(}{)},
bodyfont=\normalfont,
postheadspace=1em,
headpunct={},
qed=$\blacktriangleleft$,
numbered=no
]{solstyle}
\declaretheorem[style=exstyle]{exercise}
\declaretheorem[style=solstyle]{solution}